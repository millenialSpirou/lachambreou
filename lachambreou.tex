\documentclass{article}

\usepackage{paracol}
\usepackage{fontspec}
\usepackage{setspace}
\usepackage{csquotes}%textelp
\usepackage{titlesec}
\usepackage{comment}
\usepackage{paracol}
\usepackage[a4paper, total={4in, 8in}]{geometry}

\usepackage{dramatist}

\titlespacing*{\section}
{0pt}{5.5ex plus 1ex minus .2ex}{4.3ex plus .2ex}
\titlespacing*{\subsection}
{0pt}{5.5ex plus 1ex minus .2ex}{4.3ex plus .2ex}

\setlength{\parindent}{0.1cm}
\setmainfont[Ligatures=TeX]{ebgaramond}

\begin{document}

\section{protagoniste}
Nous racontons ici l'histoire de notre protagoniste Cédric qui oscille entre
romantisme et son anti-thèse qu'il ne peut encore formuler. Il y a des poèmes,
des faits des histoires mais surtout une contradiction entre le faire et
l'expérience. Comment trouver sa voix entre la machine rationalisée et le rêve
schizoïde. Nous commençons à un point assez central, sa dernière année de bac en
Math. Il partage un appartement rue Christophe-Colomb avec deux musiciens que
l'on pourrait aussi qualifier d'énergumenes. Il dispose d'un cercle d'amis
fidèles et variés, entouré d'un cercle concentrique de connaissances
énergisantes et divertissantes. Mentionnons Dave, qui a séduit la douce
Galiffée, et puis Joe peut-être plus tard.\\

TODO descriptions de Joe et Dave\\

\textit{Jean qui est ingénieur et fait le tour du monde, il sort d’où on sait pu
  trop, la Zambie, toujours la Zambie et la Malaysie surtout d’où il revient
  avec ses histoires abracadabrantes, une légère barbe hirsute, de nouvelles
  normes culturelles et une nouvelle personnalité qui vient se graffer sur ce
  qu’était Jean pré-nouveau voyage qui change toujours mais toujours grand et
  blond et blanc,}\\

\textit{Joe et ses lunettes rondes et son humour décapant, son charisme de dents
  tachées [...] ses cheveux gras et lisse, ses yeux sombres et son teint olive,
  ses larges poignets ses yeux olive et son regard ombrageux, son je-men-
  foutisme maintenant garni d’un concluant salaire à la radio de Radio-Canada,}\\

\clearpage


\section{marbre}

Les couleurs se versent dans leur tiédeurs ternes et l'âme de Cédric se complait
en épithètes chialeux. Le café est trop lent, il se déploie dans la tasse, comme
une routine de yogi au sourire imbécile, mielleux et perdu mais avec quelque
chose qui cloche derrière, une paix intérieure lactée et donc trouble. La
méditation n'est pas pour celui-ci, il manque de flexibilité et ne peut dont pas
s'assoir convenablement les jambes pliées. Et méditer sur une chaise, c'est con
tout de même, on dirait qu'un principe essentiel est ainsi transgressé. Et des
principes ancestraux, il en a déjà transgressés assez ces derniers temps. Dans
ce genre de mood il faut pas rester sur place, on s'active, on va faire du
sport, une bonne course dynamique pour se brasser les os et ensuite hop la
douche chaude et puis les étirements et un bon petit poisson grillé, légumes
vapeur le tout couronné d'un bon film, quelque chose de réconfortant. \\--- ---
ou l'on fume. --- L'on fume si la morosité cynique est cause révolutionnaire; la
fuite du cliché aboutissant toujours et inévitablement en cliché, en clope et
autres symboles phalliques. Mais tout de même, après tout, il faut bien meubler
sa jeunesse. \\

Et d'ailleurs là où Cédric se trouvait, les meubles ne sont pas ce qui manque.
Ça alterne entre le contemporain lisse, le canapé ancien-régime, la bay window
entre deux vases chinois, on a droit à du granit, beaucoup de granit, et un bois
que l'on pourrait qualifier de japonais ; le rouge à lèvre recouvre
approximativement 30\% des lèvres avec goût ce qui est un ratio qui fonctionne
bien et ça indique à qui sont les drinks selon la teinte; ce qui permet de
remarquer le verre orphelin de Gallifée et de lui porter alors qu'elle contemple
paisiblement la rue McGill deux étages plus bas une cigarette à la main la
fenêtre légèrement ouverte, la fumée qui s'égare vers les bassins au bout du Vieux-Port.
\\Le granit les talons les grands verres, très grands verres à vin, tout est
brillant et cristallin, avec de légères notes complémentaires de soyeux et de
velour, la pluie est légère et sophistiquée en glissant sur les grandes
fenêtres:

\columnratio{0.65} \setlength{\columnsep}{4em}
\begin{paracol}{2}
Cédric essaie
de s'extirper de sa bulle de poête cynique par le geste; il s'empare du
verre de Gallifée et essaie de se faufiler au travers de la piste de danse
improvisée, où les gens tournent et tournent et les grands talons font
tac-tac-tac et les grands verres cling cling, il bredouille un peu, aimerait
être plus souple dans le mouvement du corps, regrette de ne pas avoir appris
une danse sociale, la salsa problement, lorsqu'il était en Amérique Latine
avant d'entamer les études supérieures, il aurait peut-être eu le sang un
peu plus convivial. Il aurait dû être comme David et accepter la vie telle
qu'elle lui a été présentée au lieu de se morfondre en aphorismes à deux
piasses.\\

\emph{Un cynisme comme une peau de lion pour cacher un amour fragile.}\\
\switchcolumn
\setstretch{1.2} \phantom{}
\small
\textit{"\textelp{}donc voilà ça a été un hiver un peu difficile pour moi au
    plan personnel, après l'histoire avec ma mère et j'avais besoin d'un peu de
    nouveau, ma job au début cà passait mais après \textelp{}"\\[1em]
    "\textelp{} C'est bon comme toune ça, tu aimes tu le hip hop progressiste,
    personellement je comprends mal l'anglais mais j'aime quand c'est engagé"}
\end{paracol}

Profitons des quelques
instants où Cédric s'avance le verre de Gallifée à la main vers
la fenêtre où cette dernière se berce au gré du vent d'automne pour faire un topo
rapide.

\columnratio{0.35} \setlength{\columnsep}{4em}
\begin{paracol}{2}
\begin{rightcolumn}
    David est en train d'emménager avec Gallifée qui est toujours aussi
    empathique et chaleureuse dans un condo à Villeray grâce à son salaire de
    consultant en \textit{art-investment}, effleurer suptilement la hanche de
    Gallifée, amicalement bien sur, (pendant que son copain Dave raconte une
    vieille histoire d'universitaire à Joe histoire qui comprends une auberge de
    jeunesse, un bateau, et une omellette, 3 batons de dynamites, quelques
    cigares et un tigre asiatique et drogue, à risque de paraître vulgaire,
    \emph{évidemment} : drogue) et tirer un sourire peut-être un peu trop gras,
    mais il n'y a pas réflexion, il s'agit de réactions rapides. \\

    Tout ceci est confus et ça ne se choisit pas les sentiments, ni ceux bien
    tendres envers Gallifée ou ceux d'envie face à la situation de David. Ce
    genre de comportements ou de sentiments n'ont pas leur place au sein
    d'amitiés profondes qui ont l'âge d'un très vieux chien, quoique disons le,
    soyons \emph{honnêtes}, Gallifée est très, très jolie\\
\end{rightcolumn}

\begin{leftcolumn}
    \setstretch{1.2} \phantom{}
    \small
    \textit{"Oui je comprends comment tu te sens pour moi aussi ça a été
        difficile l'important c'est d'être ben relax, ensuite on s'en rend
        plus trop compte et c'est d'ailleurs très plaisant une fois qu'on
        se laisse allé, bon c'est sûr que c'est intimidant mais moi après
        en avoir parlé avec ma conjointe on s'est entendu qu'au final
        c'est vraiment une question de confiance et d'honnêteté
        \textelp{}"}\\

    \textit{"Écoute depuis que j'ai passé du Bikram ou Yin, je me sens
        telllement mieux, c'est comme plus passif, ça détend tout,
        jusqu'aux orteils, et maintenant eille je suis tellement plus
        productif, j'ai même reçu un bonus\ldots\\grâce au yoga, weird
        non""Ahhh ouin, effectivement, c'est spécial"}
\end{leftcolumn}
\end{paracol}
\clearpage



\begin{center}\noindent\rule{0.5\textwidth}{0.4pt}\end{center}


Le café finit par couler, une fois la toast beurrée le matin peut
tranquillement  se résorber. On échange quelques bières dans un bar quelconque
car on est samedi après tout et on se ramasse par quelque mécanisme obscur dans
un grand immeuble vitré au vieux-port de Montréal, entre deux galleries trop
chères qui vendent plus du design graphique commercial léché que de l'art, que
l'on se retrouve à rigoler avec des petits regards admiratifs en coin ce qui
est quelque peu étrange d'ailleurs parce que David et Gallifée sont habitués à
l'endroit, pas précisément celui-ci mais son essence, son zeitgeist. Mais on ne
sort pas en ménage à trois, cela ne se fait pas, il faut comparses, bonhommie,
du léger, des personnages secondaires
à notre vie qui ont des catch phrase et ajoutent la bonne teneure de
rocambolesque, il faut \emph{symétrie} donc il y a aussi Jean qui est ingénieur et
fait le tour du monde, il sort d'où on sait pu trop, la Zambie, toujours la
Zambie et la Malaysie surtout d'où il revient avec ses histoires
abracadabrantes, une légère barbe hirsute, de nouvelles normes culturelles et
une nouvelle personnalité qui vient se graffer sur ce qu'était Jean pré-nouveau
voyage qui change toujours mais toujours grand et blond et blanc, en fait tant
qu'à y être n'oublions pas d'appeler Joe pour qu'il se joigne à l'excursion vers
le party d'amis d'amis d'amis recursifs, Joe et  ses lunettes rondes et son humour
décapant, son charisme de dents tachées démontré lors de la marche du métro
vers l'édifice; il prend la peine de s'arrêter à chaque sortie de bar
pour s'introduire dans chaque discussion avec quelque présence féminine pour
en échapper un sobriquet un sourire lorsqu'il raconte une anecdote rapide
ou pousse un compliment, dents qui n'affectent pas son charisme
car il peut se le permettre avec ses cheveux gras et lisse, ses yeux sombres et
son teint olive, ses larges poignets ses yeux olive et son regard ombrageux,
son je-men-foutisme maintenant garni d'un concluant salaire à la radio de
Radio-Canada, d'ailleurs il ne se dirige pas vers les groupes de fumeurs
que pour cruiser pendant que ses amis l'attended en sirotant une bière à
la bouteille, il en profite aussi pour discuter de sujets épars, il en maîtrise
beaucoup grâce à son boulot, toujours en train de commenter tout.\\

\clearpage

Donc on monte un ascenseur au vieux-port un ascenseur qui fait zouuu tout en
douceur avec un cockpit comme si l'on voyageait dans un tube pneumatique et
on se taquine un peu, l'atmosphère est bien détendue, on est \emph{ben cocktail}.
Ça se remarque, on se dit quand même; entre deux feintes de boxes avec
Cédric Joe craque le mirroir qui lui fait dos sur quoi la joie et la
désapprobation sont totales (car le masculin, totaux, si laid) : "Eille Joe à soir casse pas toute caliss" --- "M'en
criss on Turnn Up\footnote{Vire fous, on fait le gros party, la teuf quoi} a
soir less go" "Joe\ldots J-J, tout-doux" --- "ouais d'accord Quoii
D'AUtres".  Donc on monte dans ce tube et ça fait zouuu et on giggle entre
quelques gorgées partagées de vin blanc à la bouteille. Et l'on cogne entre
deux simagrées à cette grande porte lisse et pleine. On entre dans ce loft
mezzanine dont les deux étages donnent sur une immense fenêtre qui elle
donne sur le centre-ville illuminé et le fleuve qui s'allonge.  Bien évidemment
il y a du trap, un mobilier de jeunesse flétrie--disons fin vingtaine à fin
trentaine--riche, bon rien de dynastique mais tout de même, en 2018, le
mobilier d'une telle cohorte \emph{nécessite} le trap.
\footnote{Le trap est un style musical qui a ses origines dans le
        hip-hop du sud des états-unis. Il est marqué par de très rapides coups
        de snare en triplettes sur de larges basses lines qui ondulent sous le
        rythme de gros gras kick-drum.  Le tout est garnit alors de
        \textit{mumble rap}, un style de rap où l'artiste déploie paresseusement
        ses rhymes, lorsqu'il y en a, avec l'accent d'un ivrogne sur la codéine,
        le rythme encore en triplettes: tatata-tatata-tatata-TA.  Nous pourrions
        qualifier ce dernier style d'une série de dactyles punchés à la fin par
un anapeste moderne}\\

Le loft est situé au dernier étage d'un nouvel immeuble, les planchers de granit peut-être, on admire
le tout en se délaissant de son imperméable et en enlevant ses botillons mais
quelqu'un nous enfarge: Jean est ben trop high pour délacer ses souliers polis
ou pour avoir une quelconque appréciation esthétique soutenue qu'il se trémousse
déjà en se faisant aller les bras vers la partie plus sombre de l'endroit où le
dance floor a été méticuleusement déposé, et Joe, Joe cherche déjà les verres et
n'en a rien à foutre vraiment des bâtisses, il cherche des verres surtout pour
se chercher un verre parce que la bière ça fait pas la job et il a
judicieusement ammené un fiable 26oz de Jim Bean\\

\begin{comment}Le Jim
    Bean est un whiskey, un bourbon pour être plus précis, connu comme étant
    typiffiant de l'amérique avec un gros r sale, d'une toxicité masculine, avec
    sa bouteille nettement carrée et son petit coup de coude en fin de gorgée,
il est pas mal quand même.  Et pour le prix, pour le prix\ldots
\end{comment}

\clearpage

On est dans la cuisine, on prend place, se cherche un verre, se présente
aux divers convives qui étaient déjà présents, certains pour un verre d'eau
d'autres pour fumer sous la hoote, ou encore, comme c'est le cas de Salomé
simplement pour s'éloigner de la fête parce que déjà à cette heure pas si
tardive  ça se tortille, ça fait de la grosse poudre, ça s'ostine sur la
prochaine toune, il y a à ce que l'on peut comprendre déjà eu tout
un combat de masculinité toxique, pas aux poings mais un est parti en claquant
la porte, une histoire de poker ou d'ex on ne sait plus.\\

Alors Cédric décide d'arpenter les lieux et se déplace vers les escaliers en
évitant des conversations sur la vie, l'amour et la crise financière, les
danseurs un peu trop enjoués et finalement il peut faire l'ascension du
colimaçon en bois, celui-ci nettement québécois, du frêne recyclé on dirait, et
il arrive à un cercle de petites conversations sur les fauteuils rouges amples
mais angulaires joliment installés en ménage à trois sur le bord de la rampe. Il
faut socialiser au final, on ne reste pas entre petites cliques comme de gros
quebz salles à un party, on mingle, \emph{caliss}. On fait des rencontres
inopinées avec, évidemment, la vue majestueuse sur la deuxième moitié en hauteur
de la bay window, cette lumière colorée à travers les échancrures des grands
luminaires abstrait de glissants d'étincelles.
\\

[Note de l'auteur : dialogue émotif à ajouter]\\[1em]

\clearpage

\begin{paracol}{2}
\setstretch{1.2}

\phantom
\small
\textit{"\textelp{} Faut vraiment qu'on aille au Charlevoix cet hiver
il y a un rave avec un line up de DJ de fou mon gars. Un truc de malade.
Et ensuite BIM, on s'enfile des tartiflettes, le ricard, un bon flanc,
et on se la met bien rigo, on revient de la teuf en chien de trainaux
et tout ça va être décalquant"}\\
\switchcolumn

A sa gauche il y a une \emph{salle à poud}, la chambre en temps normal
destinée aux vacanciers
américains ou français qui déboursent quelques centaines de dollars par nuit
pour l'escapade et on rentre dans cette pièce
et en fait il y a un miroir bien positionné,
la vitre vers le haut, un miroir
sans cadre, pour gratouiller tout ce qui reste sans que ça coince dans les
craques, scratch scratch l'âme de rasoir et évidemment, lorsqu'on s'en fait
proposer une tite ligne, et qu'on est là pour relaxer, et que c'est un nom de
la politique bien connu maintenant, connu pour ses opinions plutôt radicales
gauchistes, qui vous proposent la dite tite ligne, alors on dit mais oui en fait
allons-y.\\
\end{paracol}

Alors Cédric prend place dans le cercle ou plutôt rectangle courbé
de chaises en aluminium et fait un signe de tête et un gentil "Salut".
D'ailleur juste à côté on retrouve Joe qui roucoule comme un
perroquet et fait des becs dans le coup à une animatrice
de variété autrefois connue qui a d'ailleur disparu plutôt brusquement de la
sphère médiatique Québécoise, petit fait divers intéressant bien vite résolu par
l'animatrice entre deux sniffées, elle est \emph{en thèse} , elle en avait marre
des médias et de la superficialité; elle est retournée aux études comme elle
l'explique en ce moment, en \emph{thèse} sur le poète Brézilien Carlos Drummond
Andrade et sa démarche formelle face à la langue populaire, \\ on a plus les
animatrices de variété qu'on avait\ldots




\begin{center}\noindent\rule{0.5\textwidth}{0.4pt}\end{center}

Les petites heures approchent et il se retourne à contempler la vie et Salomé,
la jeune femme avocate sincère et spirituelle qui lui fait face dans la cuisine
entre le fridge et le comptoir auquel elle est indolemment accotée. Il voudrait
lui contempler les bas-fonds de l'âme et s'y plonger, mais les heures sont
petites, ses yeux sont vitreux, la musique se fait longue et plate. Il fixe un
ustensile, n'écoute rien, ni ce qu'elle dit ni le bruit de fond constant ni les
paroles du rapper \textit{Lil-Mickey-Royce}. Il lance quelques regards autour de
lui pour constater une étrange apathie, et il faudrait percer l'air et rejoindre
Salomé ou quelqu'un quelque chose. Regards croisés, une discussion authentique?
On se voit s'ouvrir à cette belle étrangère qui nous expose un intéressant
dilemme éthique dans le droit international. Faire une vraie rencontre et
prendre rendez-vous, pour une marche sur le Mont-Royal, avec un chien, c'est
l'automne, c'est coloré. Mais elle parle dans le néant, il se retourne, plonge
sa main gauche dans un gros bol de cheetos et pendant qu'elle élabore sur la
constitutionnalité post-moderne; il se liche un à un, lentement, chaque doigt de
la main gauche. \\


Joe est probablement déjà rentré avec quelqu'un(e) il ne pourra donc pas
remonter le moral à Cédric avec quelques jokes de mononc bien tournées et des
gesticulations (c'est sa seule utilité)

Cédric s'avance le verre de vin à la main, verre toujours
taché du rouge à lèvres sobres de Gallifée, en boit un grand trait et le dépose sur
une corniche car la fenêtre est ouverte et donne sur un faux balcon. Jean et Joe
cassent quelque chose de vitré en dansant, si on peut appeler cela de
la danse à cette heure-ci, c'est plutôt un rassemblement amateur de danseurs
du ventre. David vient rejoindre Cédric à la fenêtre, lui tend une bière.
Les deux prennent une gorgé, haussent les épaules. Le premier fait à l'autre un
signe de tête. Ils sortent et descendent les escaliers.\\

Une fois
sur le trottoir de la grande
rue McGill avec ses nouveaux lampadaires chics et sa belle asphalte large et
ondulée et les commerces de luxe ils se dirigent lentement vers le port en
allumant un joint.

Arrivé à la promenade derrière à la piste cyclable ils s'avancent vers la
fin d'un pier, comme une presqu'île pittoresque. \\ Ils prennent place à un
banc, râlent contre les conneries de la vie, quelques vicissitudes partagées
malgré leurs parcours divergents. Ouvrent chacun une cannette de Old Milwaukee,
par nostalgie de l'adolescence, David humecte la colle d'un autre joint alors que
son ami s'essoufle d'un soupir mélancolique mais paisible.
\\ -- Pis Dave, tu
penses tu que ça va ressembler à ça votre loft une fois retaper pis toute\\
-- Non dude, voyons, j'ai tu vraiment l'air d'un gars qui plaque des
reproductions de
Jackson
Pollock partout \\
-- Ben non Comon jte niaise\\
-- Je sais mais ça hit fort quand même de voir du monde de même avec qui
t'as jamais eu tant que ça en commun et te dire, ben oui ce serait logique,
ce serait moi dans pas long tout ça \textelp{} Pis toi, t'a fini ta maîtrise tu
vas tu au Doc?\\
-- Je sais pas trop encore, ça pu l'air trop pertinent, j'ai l'impression
de juste ingérer des bits d'informations, style oie à fois gras\\
-- Je t'entends, même vibe pour moi quand j'ai fini par finir l'école\\

\textelp{} et au fait, maintenant que j'y pense, pour votre appart là, vous
avez pas aussi commandé le même genre de comptoir contemporain en granit
messemble\\
-- C'est pas du granit, \emph{criss}, c'est du \emph{marbre}\\

Cédric humecte maintenant le joint qui lui est repassé en le tournant entre son
pouce et son index, déposant la salive avec son auriculaire à l'extrémité du
cherry, il s'émouvoit encore un peu du paysage, urbain mais intime quand
même\ldots quelques rares passants, la lumière du port, une eau trouble et
miroitante.\\[1ex]
Il décide qu'il est maintenant impératif de séduire Gallifée; préférablement sur un comptoir.
\clearpage


\section{mensonges}

Gallifee est assise sur le comptoir de marbre, bière à la main, Cédric rêvasse,
dans un fort intérieur lointain. Il a été invité à souper par le couple
bienvennant, on est à la moitié de l'hiver\\

--- le pire, c'est le pire de calisse de mois de l'année que s'exclame Galiffee\\
Dave acquièse en prenant une gorgée.\\

L'invitation fait suite à une rumeure dans le cercle d'ami comme quoi Cédric n'irait pas
très bien ces temps ci. Célibataire, à 26 ans, on pourrait le croire plus enthousiaste face
à la vie malgré les journées brèves et frettes que le tabarnak. La valse des temps modernes
se berce cependant au rhythme de l'anxiété et notre protagoniste n'y fait pas exception.
Il essaie de lire dans les cafés; lire un bouquin, prendre des marches emmitoufflées
dans un gros kanuk qui lui descend jusqu'aux genoux. Malgré les regards et lumières
feutrées de ces endroits son appartement morose rue Christophe Colomb lui casse les pattes
dès qu'il y retourne. Ça fait contraste avec celui de ses hôtes qui est bien habité.
L'appartement est habillé de façon iconoclaste, il y a du moderne, des chaises élancées
de bois et de plastique blanc qui entourent la table à diner, derrière celle-ci, en haut
de la porte du balcon arrière quelques masques africains, après tout pourquoi-pas.\\

Le repas est fini et les assiettes sont dans l'évier, on est en mode post-communnion,
c'est l'heure de dire les vrais affaires, la bière à David traine dans ses mains, il
joue à égratigner son étiquette.\\

--- tu sais tu peux tout nous dire à fée et moi \\
--- je sais merci je sais pas trop comment expliquer comment je me sens ces temps-ci\\
--- tu m'avais pas dit que tu avais rencontré une fille cute récemment dit Galifee du
haut du comptoir\\
--- oui Jolie qu'elle s'appelle, ça lui siet bien d'ailleurs elle est mignonne\\
--- Cool babe ! c'est lfun ça, comment ça c'est passé\\
--- on était tous les deux à un concert, elle portait un beau chapeau, je me suis enfargé dedans\\
--- cutee\\

En arrière il y a du jazz qui joue, un trio piano de bill evans pas mal.\\

\clearpage

--- mais j'ai encore l'impression de déconnecter de la réalité desfois\\
--- tu fumes tu encore du weed dit David\\
--- ouais trop je sais, c'est étrange on dirait que je vois des pattern partout ça me harcèle,
comme si j'avais fait trop de maths et dans chaque recoin et crévisses de me sens je vois des patterns
--- Ouaiss on a parlé  à Joe il m'a dit que tu obsédais sur les cahiers perdus d'un mathématicien
ça l'inquiétait un peu\\
--- Ouais le pire c'est que je comprends rien aux maths de ce gars la c'est de la géométrie algébrique
surtout qu'il faisait. Et check ça, il a fini sa vie avec deux décennies d'hermite complet, sur les
dernière photos il a l'air d'obi wan kenobi. En tout cas c'était un anarchiste et pendant ces
deux décennies ou il vivait tout seul dans les pyrénnées il a continué à faire de la recherche,
c'est ce qui est beau des maths, on peut faire de la recherche juste avec son cerveau, et bref
un groupe d'étudiant à rammenner ses travaux à paris et ont a peine commencer à
éplucher de ces manuscrits.\\
--- ça sonne comme si tu cherchais la clé de l'univers là mon cedric\\
--- ouais je sais mais check quand je fume on dirait que je comprends plus encore\\

regards consternés échangés entre Galifee et David suivent\\

--- et cette Jolie tu vas la revoir quand dit Galiffee\\
--- cette fin de semaine on va prendre une bière, proche d'ou elle habite, dans Rosemont.\\

Après une bonne soirée de conversation avec ses deux amis Cédric se sent plus
groundé, il remarque qu'effectivement ses recherches sur le mathématicien
Grothendieck enfumés de THC le rende dangereusement proche d'une coupure. Il
décide donc de mettre le reste de pot dont il dispose à son appartement dans le
sac donation pour sans abri qu'il tri chaque semaine. D'habitude il ne met que
quelques vieux livres, un 10 piastres et les cannettes de bière consignées dans
le sac de plastique transparent qu'il met à la rue.





\clearpage









%été 1
\section{calme}

\subsection{nid}
\input{nid.tex}
\clearpage


\subsection{dialoguePostNid}
\input{dialoguePostNid.tex}
\clearpage

\subsection{envolée}

Comment décrire l'égarement de Cédric, face à Jolie, leur relation était
difficile à concevoir en premier lieu. Lorsqu'il y a fallu mettre mot sur
l'histoire Jolie eu une idée géniale, partenaires de cyprine, étant le liquide
gluant et réconfortant qui découle de ses lèvres du bas lorsque Cédric l'excite.
Le problème étant et restant toujours quant aux autres, est-ce qu'on doit se
garder à une seule partenenaire de ce type. Or Jolie croi à l'amour libre, sans
restreinte.\\

-- si tu en a envie un jour je n'ai pas envie que tu te retiennes pour moi\\
-- et si ça te fait mal?\\

La question à laquelle on répond en la posant. Au début du printemps, pendant
que les échanges entre Jolie et Cédric restaient plus au moins espacés il avait
rencontré Jade sur une application. Elle avait une jolie frange qui lui manquait
les yeux de peu, des hautes pommettes mais une rondeur dans l'ossature qui
invitait le calinage. Chacun des recoins de son corps laissaient place à un
galbe dans lequel on avait envie de s'éteindre lentement; poitrine et fesses
amplent, petit rictus. Elle avait 22 ans qu'elle disait, Cédric la croyait mais
à peine, au creux de ses aisselles il semblait trouver une jeunesse plus
accentuée, une certaine adolescence qui transpirait de ses pores et dans sa
voix, elle parlait rapidement, par petites bourasques. Ce qui la faisait penser
inconséquentes.\\

-- scuse moi je dis de la marde \\
-- non je suis pas d'accord, tu te perds tu vas trop vite mais tu dis pas de la marde\\
-- merci tu comprends oui desfois mes mots me perdent en chemin\\

Cédric, honnête malgré lui disons avait avoué avoir passé une nuit avec Jade. Il
avait choisi son moment. C'était vers la moitié de l'été passé. Ils avaient tous
les deux consommé un peu d'ecstasy pour aller a un concert de musique
électronique ou s'enchainaient entre autre Nicolas Cruz, des percusions
foisonnantes de basses bien syncopés avec des rap plutôt blasés ou illuminés
dépendemment du contexte, des riff de guitare qui flottent sur les lignes de
basses. C'était à la salle de concert le Métropolis aux coins de St-Laurent et
Ste catherine de Montréal, en bordure du quartier des spectacles. C'est ici que
beaucoup d'évennements d'envergure défilaient mais pour Cédric le lieu lui
rappelait tout de même un cloaque de Montréal. Les itinérants, les illuminés
apprenti prophètes s'y mélaient aux jeune trentainaires bien habillés de t shirt
et jeans pour faire la fête avec l'occasionnelle robe d'été alors que les filles
plus novices se laissaient tanguer sur leur soulier à talon hauts et essayaient
de trouver une démarche qui fasse honneur à leur petites robes serrées.\\

Jolie et Cédric étaient sorti de la salle de danse du Métropolis pour s'accorder
une petite pause de sueur et de musique, ce dernier en profitant aussi pour
s'allumer une cigarette. Il y avait une foule de quelques dizaines de personnes
devant l'entrée, d'autre fêtards faisant la pause sur le troittoir et débordant
dans la rue. Encore bien emmitoufflés dans leur euphorie artificielle les yeux
de Jolie avaient tendance à se détourner vers le vide comme mielleux et rêveurs.\\

-- il faut que je te dise un truc Jolie\\
-- ouais (les yeux encore distraits mais qui se ressaisissent)\\
-- j'ai couché avec Jade la semaine passée, je pensais que tu m'avais laissé pour de bon\\
--  TODO \\

Jolie avait commencé à afficher le sourire le plus triste qu'il ait jamais vu.

\clearpage


\section{carnets}
----------------------
\begin{center}
--donc plus souple\\
l'air\\
de ses yeux à elle qui sont\\
chez eux \& se dissipent dans\\
un automne de capuches\\
les marées sortent emmitoufflé de paix\\
et/parce que quelqun est la prêt, exprès\\
au complet, peutêtre \\
presque au moins c'est en coin\\
détendu dans une ailleur proche\\
\end{center}

les hublots qui donnent sur le monde\\
il se place sur une plage\\
tiède froide humide salée qui l'ennuie\\
des cils qui lissent le paysage\\
des récifs qui sont beaux\\
pour rien mais avec gloire\\
des goélands caves\\
de la beauté donnée à voir\\
juste assez de monde\\
c'est à dire tout le monde\\
mais différents,bien éparprillés\\
\clearpage
Jolie est partie \\
sans faire un bruit\\
Cédric s'est réveillé\\
sur l'autre esti'oreillé\\
On lui a dit de pas s'en faire\\
que quand même s'tait pas un calvère\\
\\
Le soir Pelleter du bois\\
Après avoir Usé des feux\\
Écrire une chanson pour deux\\
T'expliquer y t'aime pourquoi\\
Se mariner en Acadie \\
se baigner dans une baie\\
s'acheter une perceuse à rabais\\
gosse une adirondak le mardi\\
Chanter une chanson pour deux\\
\clearpage
--------------------
\begin{paracol}{2}
la vie
Dans un cadre de porte\\
m'ennuie\\
affaissé de moitié, fatigué\\
de rien il attent une aube\\
quelque chose qui brille un peu, mais mat quand même\\
du bleu délavé vieux jeans\\
de l'eau de lac qui décape\\
un retour au passé qu'on s'inventerait\\
si ... \\
un ailleurs de chez soi \\
qui cohère, \textit{consistent} et bien pensé\\
\\
\switchcolumn
Cette année ou une autre\\
avant que ça se disloque\\
dans'--  bric a brac du froid écorné \\
on sait pu trop comment ou pourquoi\\
parfois Cédric se force mais\\
le hifi de néon, dla cathodes des arcs
qui shine les spasmes de joies\\
un peu forcées, les colliers fleurit\\
-- trajectoire, y s' ballade dans des réflections de glitter\\
les sons les cris les jouis le pulse
des marées urbaines ou de criquets\\
dans les bar ou les bibliothèques\\
les échanges les pleurs, les crises
le laissent comme une mouette\\
des frites des frites des frites\\
des frites pis rien d'autre\\
caliss y'en revient y y retourne \\
toute scintille, \\
caliss, ça descend mais
de temps en temps ça perce\\
s'en transpirer l'oreillé s'assoupli\\

\end{paracol}
\clearpage
-------------\\
L'air , après un été emmerdant de canicule poisseuse,
une brise dans laquelle je berce un utopisme
bucolique mais tout de même mouvementé.
Le réconfort d'une amour comptatible,
en soi cohérent avec nos prédispositions
génétiques respectives ou environnementales qui viennent soit d'une horizon qui
me suivrait depuis naissance comme un coucher de soleil d'Escher, ces
prédispositions me font rêver pourtant c'est maintenant assez évident que ce
plus simple , le moins décadent fanstasme semble effroyament hors de portée. Je
connais les étapes les récits les recettes les précipices à enjamber, quelques
gensà cotoyer-- des liens à cultiver-- pour parvenir à un certain échafaud
progressivement placé sociétalement, un piedstale contre l'effroi, il ne me
manque on dirait qu'un simple assaisonnement bien équilibré de vivacité et de
conviction.

\clearpage

%automne 1
\section{ecole}
\clearpage

%hiver 1
\section{psychose}
\subsection{5am}
 Cédric est en train de finir sa soirée d'étude, donc 4am à tout casser. Il est
dans son bureau, c'est à dire la partie de sa chambre qui surplombe
christophe-colomb, avec vue sur lampadaire jaune par la fenêtre car le bureau
précède la partie chambre, cette dernière en retrait, comme pour être plus
chaleureuse. Deux gros moniteurs sur le bureau, celui-ci en V mais
perpendiculaire à la fenêtre à cadres d'aluminium. Une arche avec moulures
marque la distinction entre la chambre et le reste. \\

message texte de jade:\\
--- allo toujours reveillé :) ?\\
--- oui tu veux venir\\
--- j'ai encore fuck up\\
--- arrête de dire des niaiseries, je t'attends\\

La chambre de Cédric est un ancien double salon, en avant du côté qui donne sur
la rue, son bureau; direction nord, la porte au dos de la chaise. Jade entre, d'abord
l'appartement puis la pièce double de Cédric. Il l'a senti venir et est déjà en train
de se retourner. Pas besoin de la décrire c'est simplement Vénus aux cheveux
bruns. On a envie de crier pour la dénoncer, t'a pas le droit d'être
belle-dememe on a envie de dire.

Ils se font un calin, depuis 6 mois la relation est maintenant quasi-platonique,
après toutes les histoires d'horreurs grivoise qu'elle lui raconte comment
pourrait il

--- Cétait bien ta soirée\\
--- bof comme jai closé avec Jessie et ensuite on est allé chez Joe\\
--- hmm\\
--- pis ensuite il nous en restait plus, de la poudre fak le contact à jim s'est pointé\\
--- \ldots\\
--- scuse moi je temmerde avec mes histoires\\
--- non non c'est jusqu'il approche 5am, c'est connu 5 heures c'est les baillements\\
--- ah ouais\\
--- oui c'est convenu, biologique même\\
--- ah ben pas moi\\
--- tu veux quelque chose à boire, j'ai un fond de bière\\
--- oui stp, t a des topes?\\

Cédric lui tend le paquet de 20 mcdonald, king size, cependant il sait qu'il y a
autant de tabac que dans des régulières. La forme importe quand même, les king,
plus longues laissent plus de puff aux obssesifs. Il marche vers le coté ruelle et elle
le suit jusque dans la cuisine. Elle est faite en coin, la table longe le
mur de la porte vers le balcon; rangement à cadavres (de bouteille bien
entendu).

Ils prennent place à la table de la cuisine, collée à la fenêtre
que l'on ouvre légèrement pour s'y placer la gueule avec une cigarette.

--- desfois jade j'ai peur d'être complétement fou\\
--- ben non pour moi t'es la personne la plus sensée que je connais\\
--- j'aimerais ca que t'arrive plus tot desfois\\
--- \ldots\\
--- scuse moi, est ce que tu veux du thé?\\

Et donc cédric qui se lève et active la theiere avec un clic qui disparait
dans la nuit, l'air entre poreusement par la fenêtre, comme un échange
avec la fumée qui ne sait trop où aller

La première fois elle était venu en le rejoindre à cinq am, dès le
début. Il avait besoin de compagnie après que Jolie l'eu dit que c'était la fin
de leur relation. La première de leur nombreuse ruptures avortées Cette dernière
aimait beaucoup la MDMA, elle lui avait confié la tâche d'en acheter pour un
party ou il n'était pas allé. Cédric avait alors une dizaine de gellules. Il
avait décidé de commencer tout seul à les enfiler
une par une, histoire de geler le deuil. Comme si déconnecter pour ces trois
jours permettrait une reconnection ultérieure. En cette fin d'été lorsque Jade
était arrivée à 5 heure il lui avait demandé si elle en voulait elle avait
répondu oui et ils s'étaient donc fait un party à deux pour quelques jours.


Maintenant les deux en fins de soirées respectives exaspérés par la vie.
Jade travaille pour une agence d'escortes réputés, cosmopolitan de son
nom de site web, elle se voit dans l'écran de cédric lorsqu'elle lui raconte
l'histoire de son embauche.

\clearpage
Cédric allongé sur le ventre, Jade sur le dos, cote a cote le silence valse,
bientot le telephone cellulaire de Jade sonnera; elle le prendra vivement,
déclarant que c'est son ange gardien qui l'appelle toujours à 5h30, une fois que
son shift fini


--- je sais que je te ferai pas arreter, ta job je veux dire\\
--- merci\\
--- mais si mettons, je sais pas, tu voudrais pas etre serveuse a la place\\
--- j'aime bien m'occuper des gens ta raison\\
--- et je sais, que c'est pas juste, unidirectionnel, je sais pas comment dire\\
mais, c'est plus toi qui me sauve ces temps ci\\
--- faut que t'arrête de penser la\\
--- mais j'aimerais tellement ça tsé,\\
--- ouais je sais\\
--- te sauver\\
--- tu sais j'hais pas ma job tant que ca
--- est ce que ta un bon driver au moins
--- ah ouais ye super vraiment


Un arbre entre le lampadaire et la fenêtre de la chambre, ainsi la lumière
tapisse l'endroit en valsant légèrement. Ils se sont parlés la première fois sur
une application de rencontre l'été passé, on est maintenant en novembre. Dès le
départ Cédric était comme fier de voir Jade au dela de son physique. Elle parle
de façon désordonnée, comme si tout devait jaillir en même temps. Les moins
perspicaces ne voient pas toute la beauté derrière ces mots, comment si ils
prenaient leur temps de se traduire en cohérence il y aurait une poésie qui ne
se traduis pas dans ces mots comme épeurés de sortir trop de vérités\\


Le cellulaire de Jade qui sonne; cette dernière: c'est mon ange gardien!
ce dernier reste anonyme pour Cédric. Elle répond, c'est une voix à accent
Africain qui répond. Selon Jade il est en ce moment en Allemagne; il fait partie
d'une équipe d'arts martiaux qui le fait voyager. Le téléphone est mis en mode
appel conférence

``Je suis chez le gars qui a appelé la cops sur moi''\\
`` ah ouais! vraiment !''\\
``mais Yo ella m'a raconté des histoires qui font peur!''  que doit rétorquer\\
``Cédric, mettre les faits à plat''\\
On entend un rire qui crépite du haut parleur du téléphone de Jade.

\clearpage

\subsection{bresil}
Jade et Cédric marchent sur la rue St-hubert. Arrivés au coin Beaubien, par là
que les commerces commencent à border la rue, ils croisent Fernando et Carlos.
Le premier de Guinée, le deuxième du Brésil, ils demandent quelque information
de touriste; comme où aller prendre un verre. Cédric voit la une occasion de
pratiquer son portugais; langue qu'il avait entâmé d'apprendre après son
décrochage d'études polytechniciennes d'ingénieur, il invitent donc ces deux
nouveaux acolytes à les suivres au Notre dame des quilles, établissement réputé
pour son ouverture d'esprits et ses tendances alternatives.\\

Ils marchent 2 a deux, largeur du trottoir obligeant. Cedric et Fernando
discutent litterature, ce sont deux programmeurs d'ordinateurs dont la veritable
passion est la litterature, c'est une révélation pour eux deux de se retrouve si
proche mentalement ainsi que geographiquement, malgre les continents qui les
séparent.\\

Lorsqu'ils arrivent au bar ils prennent place au comptoir en L. Fernando et
Cedric continuent leur discussion littéraire alors que Carlos et Jade s'effacent
sur le trait inférieur du L. Ces deux premiers partagent la même idée de la vie,
écrire du code informatique parce que ça se vend, alors que la poésie; personne
ne paye pour celà.\\

Cédric surveille du coin de l'oeil Jade et Carlos, il a l'air, sinon de la
dérenger d'être au moins, irritant. Elle se lève après quelque dizaine de minute
pour venir jouer avec Cédric, comment le fait-elle? Et bien elle a l'air d'aimer
lui mettre les mains dans la figure, se retourne danse dos à ventre sur lui,
bref, des simagrées. Cédric continue tant bien que mal sa conversation avec
Fernando\\

En sortant Jade précise à son ami platonique qu'elle n'aime pas le
compère Brésilien, il y a une note dans sa voix qui trahi comme une
espèce de connaissance de l'individu que l'on aurait pas prédit.\\

Le groupe se resepare deux a deux, on se promet de se revoir;
pour ce faire cedric a ajoute fernando comme ami sur facebook.
On voit ainsi qu'il travaille pour la fondation tomas sankara et
pour linux international, un peu de googlage serverait bien; mais
a premiere vue il s'agit la d'un organisme visant a favoriser l'education
sur l'informatique en afrique de l'ouest.\\




\clearpage

%été deux
\section{confinnement}
\subsection{lappart}
cédric a 26 ans et vit tout seul dans un bon gros 4 et demie de biais au marché
atwater par quelques pâtés de maison. Il se molasse de jour en jour à écouter du
jazz et lire les livres de sa bibliothèque du salon qu'il n'avait pas encore lu.
Le bureau était disposé dans sa cuisine avec un ordinateur et deux speakers, un
clavier et une track ball ``comme une souris mais où on fait rouler une grosse
balle sur elle même pour faire bouger le curseur.'' Il fait une diète, histoire de
se débarasser de quelques dizaines de kilos de trop qu'il avait accumulé pendant
l'hiver, un des effets des antidépresseurs qu'il prend chaque matin avec un
espresso très allongé. Il était maintenant passé au Prozac, quelque chose de
plus mainstream que son ancienne médication; il la choisissait lui-même avant,
appelait son psychiatre sa pharmacienne et dans toute chose incluant celle-ci,
il faut dire que Cédric était hipster à cette époque.

celà fait maintenant 6 mois que Jolie l'a quitté, elle avait ses raisons,
des bonnes, un soir elle lui avait dit:\\
--- c'est parce que la vie ne dure pas indéfiniment ce qui confère à chaque moment une valeur infinie\\
--- et moi ceux qui disent ça j'aimerais bien savoir à quel age ils décideraient que ce serait l'heure du suicide, si nous étions immortels
--- mais on est pas immortel\\
--- en tout cas moi si je l'étais j'en serais pas mécontent, existentiellement parlant

\begin{comment}
\end{comment}












\end{document}
